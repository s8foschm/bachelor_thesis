%!TEX root = ../Thesis.tex
Despite being a central focus in biological and medical research, cancer is still a leading cause of death in the entire world~\cite{owid-causes-of-death}. Challenges for cancer treatment include its heterogeneity and the unpredictability of drug response it implies. Ideally, cancer treatment could be customized for every patient individually, based on the biochemical profiling of their specific cancer. Finding statistical connections between cellular properties of cancer cells and their drug response is one of the many approaches currently used in cancer research, since it is a less powerful, but more achievable strategy.\\
This thesis aimed to find statistical links between biological properties of cancer cell lines, such as gene expression alterations, and the sensitivity of those cell lines to cancer drugs.
%For this, we used the GeneTrail~\cite{genetrail} platform, which implements 
For this, we use Gene Set Enrichment Analysis (GSEA), which was originally introduced by Subramanian et al.\ in 2005~\cite{gsea_original}.
%More specifically, we used the unweighted variant of the GSEA algorithm.
This method aims to mathematically assess whether, in a sorted list of biological entities (e.g.\ cancer cell lines, sorted by sensitivity to a specific drug), the entities having certain biological properties are accumulated at the beginning, at the end of this list.
%This approach enables us to gain deeper biological insights than those immediately evident from the data.

\section{Data}\label{sec:data}
For this thesis, we used cell line data from the genomics of drug sensitivity in cancer (GDSC) database~\cite{gdsc,gdsc_paper}. Cancer cell lines are essential in cancer research due to their cost-effectiveness, reproducibility, and ease of maintenance, enabling the study of cancer biology and drug responses. They are very well-characterized and allow researchers to investigate diverse cancer types despite having limitations, such as lacking the tumor microenvironment and immune interactions seen in vivo. Additionally, prolonged culture can result in genetic drift, reducing their relevance to human cancers.\\
We used the GDSC to provide drug sensitivity and gene expression values for cancer cell lines.
Additionally, we used pathway data taken from the KEGG database~\cite{kegg_website,kegg_original_paper}.

\section{Approach and Results}\label{sec:approach_results}
We performed three different enrichment analyses. In the first one, namely the \textbf{gene-based drug sensitivity enrichment}, we performed an enrichment analysis with cell lines sorted by drug sensitivity values (IC50 and CMax-Viability) as the sorted list, and gene expression alterations as the categories.
%The drug sensitivity values came from the GDSC database, and the categories were calculated previously based on gene expression values also coming from the GDSC database.
%The analyses were performed for both the GDSC1 and GDSC2 datasets, as well as for IC50 and for CMax-Viability.
We identified a number of genes which were found to be significantly associated with increased drug sensitivity or resistance across many drugs, and looked for biological links of these genes to cancer drug sensitivity and the Hallmarks of Cancer. Five genes were discussed in detail.\\
%The genes considered included \textbf{TIP1}, \textbf{NLRC3}, \textbf{PTPRC}, \textbf{IL2rg} and \textbf{ARHGAP15}.\\
In our second enrichment, the \textbf{pathway-based gene expression enrichment}, we performed GSEA with gene expression values and gene expression pathways provided by the KEGG pathway database. Here, we determined the expression of pathways as a whole, assessing whether certain pathways were deregulated in specific cancer cell lines. We did not use the expression values themselves, but normalized them using absolute p-values. This was done because the up- or down-regulation of certain genes can have a multitude of effects on pathway activity, specifically regulating it in any direction due to e.g.\ positive or negative feedback loops and the complex interaction between individual genes in a pathway or regulatory network~\cite{systems_biology_biological_circuits}. If a gene is part of a pathway, its up-regulation can just as well increase the expression of the pathway, as it can decrease it. For this reason, by considering the absolute z-scores, we only considered deregulation of pathways and did not consider the regulation direction.\\
Performing enrichment analyses considering pathway activity measures instead of simple deregulation scores using absolute z-scores could allow for an improvement of this enrichment analysis. These scores, as proposed by Hui et al.~\cite{pathway_activity_inference}, would allow us to get more reliable scores for pathway activity, as well as reintroduce the regulation direction into consideration.\\
%Six pathways were discussed in detail.\\
%The pathways we found to be significantly deregulated include the \textbf{neuroactive ligand receptor interaction} pathway, the \textbf{tyrosine} and \textbf{retinol metabolism} pathways, the \textbf{viral interaction with cytokine and cytokine receptor} pathway, as well as the \textbf{chemical carcinogenesis} pathway and lastly the \textbf{drug metabolism --- cytochrome p450} pathway.\\
In our third and final enrichment analysis, the \textbf{pathway-based drug sensitivity enrichment}, we used the results of the previous enrichment analysis to generate categories. These were similar to the first enrichment, containing cell lines where the expression of specific pathways was altered, with the difference that we didn't consider the regulation direction by using absolute p-values for normalization. With these new categories, we found pathways which were significantly associated with a change in drug sensitivity when deregulated. We found and discussed three pathways of which the deregulation was associated with an increased drug resistance for many drugs.
%included the \textbf{primary immunodeficiency (PID)} pathway, the \textbf{B-cell receptor signaling pathway} and the \textbf{FC-Gamma receptor-mediated phagocytosis} pathway.

We then performed an exemplary analysis, focussing on \textbf{Rapamycin}, a drug which is commonly used for cancer treatment and other clinical applications. We considered the results of the two drug sensitivity enrichment analyses, focusing on the most significant genes and pathways, and trying to find biological associations between these genes and pathways and the mechanism of action of the drug itself.\\
We found three genes to be significant, and explained their function and found a potential link to the mode of action of Rapamycin for each of them.\\
%, namely \textbf{CtBP2}, \textbf{FAM129B} and \textbf{Myo1b}. For all of these genes, we explained their function and found a potential link to the mode of action of Rapamycin.\\
For pathways, we also found three significant ones, where we also explained their function and found a potential link to Rapamycin actions, confirming these results.
%These were the \textbf{Intestinal immune network for IgA production}, the \textbf{hematopoietic cell lineage} and the \textbf{Cell Adhesion Molecules (CAMs)} pathways. For these, we also found potential links to the mode of action of Rapamycin, confirming these results.

We also performed two \textbf{PCA} analyses. The first one was on the results of the second enrichment, reducing the dimensionality of the pathways to two in order to be able to create a two-dimensional scatter plot.
%The matrix on which the PCA was performed contained p-values for combinations of pathway expression $\times$ cell lines.
%Each dot in this plot represented a cancer cell line, and was colored according to the site or histology of that cancer cell line.
We found that there was a strong separation in the gene expression profiles between fluid tumors (hematopoietic or lymphoid) and solid tumors (others), which can be explained biologically by the different environments faced by these tumors and the requirements the environment poses on the tumor.\\
We then hypothesized that the differences in gene expression between fluid and solid tumors could also translate to differences in drug response.
To verify this claim, we performed one more PCA, this time considering drug sensitivity values directly.
%matrix of CMax-Viability values for drugs $\times$ cell lines.
In the scatter plot we created from the results of this PCA, we observed that the cell lines indeed also separate based on their drug sensitivity.

In summary, we were able to set most of our results into a biological context and found them coherent overall. While this is a good sign for the validity of our method, finding already known results is of very limited use. However, we did also find genes or pathways which were previously not known for any effect on drug response. Our method could therefore also be considered to identify new cancer drug response markers.

\section{Limitations and Outlook}\label{outlook}
While our approach is already promising, it can be extended and improved upon in several ways. Such improvements could concern the algorithmic aspect, the post-processing and biological interpretation aspect, as well as the data aspect.\\
%Considering the input data, the improvement in dataset from GDSC1 to GDSC2 was already a significant step. More reliable input data, such as gene expression, drug sensitivity or pathway data, are always very desirable for bioinformatics reserarchers.\\
One idea for improving this approach from a data perspective would be to use data from real patients. Decent alternatives would be to use xenografts~\cite{xenograft} or organoids~\cite{organoid}, which are both in vitro models with great potential in oncology. Considering real human patients, the ability to perform enrichment analysis on live data could have the potential to be a very powerful tool in treatment strategy and drug selection.\\
Instead of using gene expression properties (altered genes or pathways) for the categories, considering other omics datatypes could also generate potentially interesting insights. For example, mutations or copy number alterations (CNAs) are very typical in cancer and could potentially be used to characterize cancer cells~\cite{hallmarks-of-cancer}.\\
We considered the CMax-Viability as a rather recent and little-used measure of drug sensitivity~\cite{cmax_viability}. As opposed to IC50 values, CMax-Viability allows for the comparison of values between drugs. For this reason, one could also perform an enrichment analysis where, instead of sorting cell lines for each drug, a sorted list of drugs for each cell line would be used. By using properties of the drugs, we could gain insights about drug response from the perspective of drug properties instead of properties of the cell.\\
We found some signs for a potentially significant difference in drug response between liquid tumors (hematopoietic and lymphoid) and solid tumors. This was observed and hypothesized previously~\cite{fluid_solid_tumors_sensitivity, tissue_specificity_of_in_vitro_drug_sensitivity}. To investigate this link further, one could for example perform enrichment analyses focused solely on data from liquid tumors, or on data from other tumors having one single common histology.\\
The algorithmic aspect can be improved upon from an efficiency perspective: currently, both the running sum statistic as well as the precise p-value computation algorithms are heavy on both computational power and runtime. An implementation allowing to run multiple running sum statistics at the same time would be desirable, since it would somewhat raise the multi-threaded performance and memory requirements, but significantly slash the execution time of the running sum statistic, since running a single running sum statistic is not especially heavy.
%From the post-processing and interpretation point of view, there is also potential for improvements. In the current bioinformatics and computer science landscape, an obvious choice would be to attempt to apply modern machine learning approaches and methods to the interpretation of the p-values. These could potentially provide a much greater and more profound understanding of the underlying structure and patterns in the incredibly large quantity of p-values, which are impossible to grasp for humans.

To conclude, the approach of using gene set enrichment analysis to identify biological properties associated with drug response shows promise, and its results could be used in different ways in cancer research and personalized cancer therapy. Firstly, they can give insight about links between biological properties like gene expression and drug response. Secondly, these results could be used for further study: by using the identified alterations as features in a machine learning model predicting drug response~\cite{drug_sensitivity_prediction_ml_benchmarking}, we could potentially predict better drug sensitivities. Lastly, knowledge about the influence of gene expression alterations and other molecular properties gained from enrichment analysis could be used for the evaluation of drug candidates in clinical cancer treatment.