Despite being a primary focus of medical and biological research for decades, cancer treatment continues to be one of the most pressing unresolved challenges.
It is characterized by its prevalent heterogeneity, which inherently complicates any cancer treatment. Personalized medicine aims to make educated decisions about treatment strategies and drug selection based on the individual characteristics of a tumor. Understanding what biological properties influence the drug response for specific tumors and drugs is indispensable for cancer treatment, but also for the more fundamental understanding of cancer and the development of new drugs.

In this thesis, we aim to identify potential drug response markers for cancer cell lines, taken from the Genomics of Drug Sensitivity in Cancer (GDSC) database. To this end, we use the Gene Set Enrichment Analysis (GSEA) algorithm to perform three seperate enrichment analyses, each of which approaches the characterization of cancer and the identification of new markers from a slightly different angle. GSEA aims to assess whether a certain biological category is especially enriched or depleted in a sorted list of entities.\\
%We discuss the methodology behind GSEA, including the running sum statistic and the different methods for the computation of p-values, in detail.\\
We use GSEA to identify gene and pathway expression alterations associated with a change in drug sensitivity. We identify both drug-agnostic and drug-specific markers.

For most identified markers, we were able to find literature evidence or biological explanations that link them to drug sensitivity or resistance mechanisms, which validates our method.\\
Our method could potentially be used to identify novel, previously unknown markers.