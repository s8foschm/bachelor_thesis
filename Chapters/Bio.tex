\section{Biological features}
\textsc{To-Do.}\todo{find a better title}

\section{Hallmarks Of Cancer}
\todo{introudction to hallmarks}The first of these hallmarks is that cancer cells are self-sufficient in growth signals. That means that they don't require signals (by way of hormones, e.g.) to initiate cell growth; they either produce these signals themselves (through so-called \textit{autocrine signaling}), activate the growth pathway (that would usually be activated by the previously mentioned signals), or destroy the negative feedback loops in place to prevent excessive growth through these signals.\\
The second and third hallmarks are that cancer cells tend to be insensitive to anti-growth signals and apoptosis. Similarly to the previous hallmark, the cancer cell circumvents the typical mechanisms meant to avoid uncontrolled growth and abnormal proliferation of cells; or, in the case of apoptosis, the most drastic and last line of defense against abnormally proliferating cells, the induced cell death.\\
The next two hallmarks are related to the actual uncontrolled growth and multiplication of tumors and cancer cells. They are, in fact, the ability of cancer cells to replicate an unlimited number of times (whereas regular cells die after a certain number of divisions), as well as the sustained formation of new blood vessels (\textit{angiogenesis}) to ensure the needed supply of oxygen and nutrients for the growth of a tumor.\\
Finally, the last hallmark of cancer is the ability of cancer cells to break away from their original site or organ of origin to invade other surrounding tissue and spread all across the body in a process called \textit{metastasis}. This is the most known consequence of cancer, and extremely dangerous for the health and survival of the patient, as it can lead to failure in any organ that it might reach, and metastatic cancer (also known as \textit{Stage IV}) can currently not be treated.

As originally introduced by Francis Crick in 1958, the \textit{Central Dogma Of Biology} states that genetic information is passed down in a flow, starting as information stored in form of \emph{DNA}, which is then \emph{transcribed} into \emph{RNA}, which is in turn then \emph{translated} into \emph{Proteins}, in a process called \emph{protein biosynthesis}. The individual processes of \emph{transcription} and \emph{translation} are very complex, involving intricate molecular machinery and regulatory mechanisms to orchestrate gene expression and protein synthesis.

In isolation, DNA is a very stable and reliable way of storing information. However, with many processes occurring within the cell at all times, such as DNA reparation mechanisms, DNA replication or DNA transcription, there is always the possibility of errors. Such errors are one way to obtain \emph{mutations}, a modification of the DNA, changing the information stored. Those changes can have consequences, changing biological aspects and processes in the cell that are controlled by the DNA. From the \emph{Central Dogma Of Biology}, one can clearly see how changes to the DNA will inevitably provoke changes in the biology of the cell. Such changes can, for example, include the up- or down-regulation of the expression of certain genes. In some cases, these mutations might accumulate, which can facilitate the development of cancer~\cite{stem_cell_mutations}.

\color{gray}One main challenge in cancer therapy is that cancer cells and their properties tend to differ strongly between different tumors, \color{black}since the mutations which cause the cancer cells to differ from regular cells are inherently random.
%Therefore, there is a vast and diverse array of mutations\todo{not only mutations, but their consequences, in my case changes to gene expression profiles}, which can occur in a very varied manner between different tumors or patients.
Considering the \emph{Central Dogma of Biology}, these mutations clearly change different biological aspects in a cell, such as gene expression (among other things like proteins being modified or misformed), which causes them to create a vast and diverse array of cells with different biological properties between different tumors or patients.\\
Still, some similarities and common traits can be identified between different tumors, as first done in 2000 in \textit{The Hallmarks of Cancer} by Hanahan and Weinberg~\cite{hallmarks-of-cancer} and later extended in 2011, now comprising ten characteristics that are hypothesized to be common among all human cancers~\cite{hallmarks-of-cancer-next-generation}.\\
These specific cancer-facilitating modifications can be summarized as each bypassing one cancer defense mechanism deeply engrained into the fundamental biology of each and every cell. For example, one of these hallmarks is the ability of a cell to evade apoptosis~\cite{hallmarks-of-cancer}. Quite clearly, the controlled killing of cells by apoptosis is an important part of the cancer defense machinery in a cell~\cite{apoptosis_in_cancer}; so, to be able to proliferate as a tumor, it is important for a cell to be able to evade apoptosis. Two other proposed hallmarks are the ability to be self-sufficient in growth signals~\cite{hallmarks-of-cancer}, as well as the insensitivity to anti-growth signals~\cite{hallmarks-of-cancer}. In a typical healthy cell, cell growth is strictly regulated, but for a cancer cell which wants to grow and proliferate as much as possible, it needs to be able so self-supply the signals needed to grow, as well as ignore the signals intended to make the cell stop growing.
%\color{gray}These specific\todo{don't describe the hallmarks this abstractly, use some concrete examples to intuitively understand what Hallmarks are and do.} cancer-facilitating modifications can be cataloged into different fundamental ways of function, each bypassing one cancer defense mechanism deeply engrained into the fundamental biology of each and every cell. Six fundamental hallmarks were identified, as well as two emerging hallmarks and two enabling characteristics, which are suspected to lead to the development of the proposed hallmarks\todo{detail level is bad. Irrelevant if they are "hallmarks" or "enabling characteristics", what's relevant is the way they function}.\color{black}

\section{Biological Pathways}
\textsc{To-Do}.