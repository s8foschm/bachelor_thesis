%!TEX root = ../Thesis.tex
%\textsc{Basic structure:
%\begin{itemize}
%    \item Central Dogma Of Biology
%    \item Mutations + Consequences (Gene Expression, ...)
%    \item that facilitates formation of cancer (get into Hallmarks)
%    \item similarities (Hallmarks), but still always very heterogeneous
%\end{itemize}
%}

Cancer is a family of diseases which has been at the forefront of biological and biomedical research for many years now. It is characterized by abnormal cell growth and multiplication, forming clumps of cells called \textit{tumors}, which can spread and invade different parts of the body and damage organs, often leading to death~\cite{who-cancer}. It has been the second most important cause of death globally in 2023~\cite{owid-causes-of-death}.\\
%\color{gray}Still, scientists lack a profound understanding\rewrite{positiver gestalten: viele Durchbrüche, aber trotzdem viele unklar} of the causes, development and consequences of cancer, and comprehensive success in cancer therapy has not been achieved yet, as most current treatments remain highly dangerous, invasive, or their success rate very uncertain~\cite{side-effects-cancer-treatments, narrative-review-heterogeneity-challenges}.\color{black}\\
In the last few decades, significant progress has been made in the research into cancer, its origins, formation, processes and potential therapies. Despite that progress, a profound and extensive understanding of cancer, as well as comprehensive success in cancer therapy, have not been achieved. Currently, most cancer therapies remain higly dangerous, invasive, or their success rate very uncertain~\cite{side-effects-cancer-treatments, narrative-review-heterogeneity-challenges}.
% side-effects-cancer-treatments: yale medicine fact sheet about side effects
% narrative-review-heterogeneity-challenges: paper on ncbi about heterogeneity, highlighting (among other things) the unpredictability
% main causes of death (in Germany and) all over the world

%\textsc{1-2 sentences on Cancer in genral, or really start on Central Dogma of Biology?}\\
As originally introduced by Francis Crick in 1958, the \textit{Central Dogma Of Biology} states that genetic information is passed down in a flow, starting as information stored in form of \emph{DNA}, which is then \emph{transcribed} into \emph{RNA}, which is in turn then \emph{translated} into \emph{Proteins}, in a process called \emph{protein biosynthesis}. The individual processes of \emph{transcription} and \emph{translation} are very complex, involving intricate molecular machinery and regulatory mechanisms to orchestrate gene expression and protein synthesis.\\
%\todo{use the Dogma! it is important here because it visualizes the consequences of mutations immediately}
In isolation, DNA is a very stable and reliable way of storing information. However, with many processes occurring within the cell at all times, such as DNA reparation mechanisms, DNA replication or DNA transcription, there is always the possibility of errors. Such errors are one way to obtain \emph{mutations}, a modification of the DNA, changing the information stored. Those changes can have consequences, changing biological aspects and processes in the cell that are controlled by the DNA. From the \emph{Central Dogma Of Biology}, one can clearly see how changes to the DNA will inevitably provoke changes in the biology of the cell. Such changes can, for example, include the up- or down-regulation of the expression of certain genes. In some cases, these mutations might accumulate, which can facilitate the development of cancer~\cite{stem_cell_mutations}.

The mutations causing cancer might occur during the lifespan of a cell (for example, caused by mutagens reacting with the DNA, or by an error in DNA replication)~\cite{mutations} or be inherited from previous cells\addref{do I need a source that mutations are passed on through mitosis? maybe mention \textbf{mitosis} explicitly}. Often, it is not a single mutation which causes a cell to become cancerous, but the accumulation of different mutations~\cite{stem_cell_mutations}. Certain substances or exposures, known as \textit{carcinogens}~\cite{carcinogens-def}, such as for example tobacco, different viruses or some types of radiations, can greatly increase the probability of a cell developing cancer~\cite{carcinogens}. Other common carcinogens include obesity, alcohol consumption, poor diet or lack of physical activity~\cite{carcinogens,carcinogens-def}. The probability of developing cancer can be reduced by avoiding known risk factors, but still, the formation of the disease is random and can therefore never be completely ruled out.

%Often, cancer formation is triggered by mutations in the DNA of the affected cells. These can either be inherited or happen during the cells lifespan\addref{during the cells lifespan}. Certain substances or exposures, known as \textit{carcinogens}~\cite{carcinogens-def}, such as for example tobacco, different viruses or some types of radiations, can greatly increase the probability of a cell developing cancer~\cite{carcinogens}. Other common carcinogens include obesity, alcohol consumption, poor diet or lack of physical activity\addref{lack of physical activity}. In summary, the probability of developing cancer can be reduced by avoiding known risk factors, but still, the formation of the disease is random and can therefore never be completely ruled out.
One main challenge in cancer therapy is that cancer cells and their properties tend to differ strongly between different tumors, since the mutations which cause the cancer cells to differ from regular cells are inherently random.
%Therefore, there is a vast and diverse array of mutations\todo{not only mutations, but their consequences, in my case changes to gene expression profiles}, which can occur in a very varied manner between different tumors or patients.
Considering the \emph{Central Dogma of Biology}, these mutations clearly change different biological aspects in a cell, such as gene expression (among other things like proteins being modified or misformed), which causes them to create a vast and diverse array of cells with different biological properties between different tumors or patients.\\
Still, some similarities and common traits can be identified between different tumors, as first done in 2000 in \textit{The Hallmarks of Cancer} by Hanahan and Weinberg~\cite{hallmarks-of-cancer} and later extended in 2011, now comprising ten characteristics that are hypothesized to be common among all human cancers~\cite{hallmarks-of-cancer-next-generation}.\\
These specific cancer-facilitating modifications can be summarized as each bypassing one cancer defense mechanism deeply engrained into the fundamental biology of each and every cell. For example, one of these hallmarks is the ability of a cell to evade apoptosis~\cite{hallmarks-of-cancer}. Quite clearly, the controlled killing of cells by apoptosis is an important part of the cancer defense machinery in a cell~\cite{apoptosis_in_cancer}; so, to be able to proliferate as a tumor, it is important for a cell to be able to evade apoptosis. Two other proposed hallmarks are the ability to be self-sufficient in growth signals~\cite{hallmarks-of-cancer}, as well as the insensitivity to anti-growth signals~\cite{hallmarks-of-cancer}. In a typical healthy cell, cell growth is strictly regulated, but for a cancer cell which wants to grow and proliferate as much as possible, it needs to be able so self-supply the signals needed to grow, as well as ignore the signals intended to make the cell stop growing.
%\color{gray}These specific\todo{don't describe the hallmarks this abstractly, use some concrete examples to intuitively understand what Hallmarks are and do.} cancer-facilitating modifications can be cataloged into different fundamental ways of function, each bypassing one cancer defense mechanism deeply engrained into the fundamental biology of each and every cell. Six fundamental hallmarks were identified, as well as two emerging hallmarks and two enabling characteristics, which are suspected to lead to the development of the proposed hallmarks\todo{detail level is bad. Irrelevant if they are "hallmarks" or "enabling characteristics", what's relevant is the way they function}.\color{black}

The different mutations and other biological modifications (such as gene expression alterations) present in different tumors cause a significant heterogeneity in cancer. They can differ in multiple aspects between different patients (\textit{inter-patient heterogeneity}), between different tumors of the same patient (\textit{inter-tumor heterogeneity}) or even between individual cells within the same tumor (\textit{intra-tumor heterogeneity})~\cite{heterogeneity-implications-targeted-therapeutics}.

One of the problems caused by this heterogeneity is that it makes the way a certain tumor or patient would react to a certain type of cancer therapy very unpredictable. A therapy strategy which has been clinically proven to be effective for one patient might not do anything for another patient with the same kind of tumor. It is a factor which significantly limits the efficacy of targeted therapies currently~\cite{heterogeneity-is-it-a-problem}, which makes it a very interesting topic of research for optimizing the ways we treat cancer patients. Clinically, the frequency at which cancer cells evolve and change, as well as the intra-tumor heterogeneity pose a large challenge: They might induce a significant difference between, on the one hand, the cells on which the genetic profiling has been performed after they were obtained from a biopsy, and, on the other hand, the cells which are in the tumor at the time when the therapy is applied~\cite{heterogeneity-is-it-a-problem}\todo{interesting, but not really relevant/analyzable in my specific case}. That difference makes the efficacy of a treatment unpredictable in many cases~\cite{narrative-review-heterogeneity-challenges}, even if that treatment might have been specifically selected due to the profiling of the specific tumor.

Current methods for treatment of cancer are very varied. One common therapy is using chemotherapy (toxic compounds to kill cancer cells) or radiation therapy (using x-ray, gamma rays or similar types of radiation to kill cancer cells). The problem is that these methods are very aggressive and risky, since they are unable to filter for cancer cells and also attack regular, healthy cells, making these kinds of therapy very taxing on the health of the patient. Chemotherapy is one of the most commonly used measure against cancer currently, but it has, for the most part, a drastic impact on the patients health and quality of life, as it can cause many side effects such as poor appetite due to nausea and vomiting, pain, fatigue, as well as psychological symptoms caused by aspects such as chemotherapy-induced hair loss~\cite{cancer-chemotherapy-and-beyond}. There are also more targeted methods of cancer therapy being developed and applied, which do limit the side effects of chemotherapy since they are not as broadly cytotoxic, but they are not yet available for all cancer types and come with the caveat that the efficacy of these therapies on different patients is going to be very unpredictable~\cite{cancer-chemotherapy-and-beyond}.

Personalized cancer therapy is one of the main goals of cancer research, since it would allow medical practitioners to administer drugs with a higher probability of success and a lower chance of unwanted side effects~\cite{meric2012overcoming}. The dream scenario would be the ability to produce tailor-made drugs for every patient, but that is still not available for all types and subtypes of cancer~\cite{personalized_cancer_medicine}.\\
%very important point here, because this is what my thesis is all about. Expand and reference biological/molecular properties mentioned earlier
In recent years, there has been a shift from an organ-centric approach to cancer treatment to a personalized, patient-specific approach, based on deep molecular analysis~\cite{personlized-medicine-recent-progress-cancer}. This has been one of the most important advances in modern oncology~\cite{personlized-medicine-recent-progress-cancer}. Currently, molecular profiling is increasingly being used as a guiding principle for cancer therapy, but the current landscape of the scientific literature is vast and overwhelming. Tools like the \textit{Personalized Cancer Therapy} Knowledge Base~\cite{personalized-cancer-therapy} are being created for that purpose; but still, the necessary information for the specific situation of a patient might be difficult to find or even unavailable. Biomarker-guided medicinal product development has been included in the 6th and most recent revision of the \textit{Guideline on the clinical evaluation of anticancer medicinal products}~\cite{ema_guideline}, published by the European Medicines Agency in January 2024. Still, the processes of changing official regulations and recommendations can often take a very long time, which is why the previously mentioned \textit{Personalized Cancer Therapy} Knowledge Base aims to provide scientific support for the implementation of personalized cancer therapy beyond the official regulations and recommendations~\cite{personalized-cancer-therapy}.

Due to the very significant variance and heterogeneity observed in naturally occuring cancer cells, a model system is needed for research purposes. A very commonly used in vitro model system in cancer research are cancer cell lines. These consist of highly reproducible cancer cells kept in a lab setting, with very well-known properties~\cite{cancer_cell_line_definition}. Cancer cell lines are very useful because they are very well-characterized with omics technologies (i.e.\ genomics, transcriptomics, and proteomics)~\cite{cancer_cell_lines_useful_model}. They are able to provide an indefinite source of biological material for experimental purposes~\cite{cancer_cell_lines_useful_model}, which allows for experiments to be reproducible. In addition, the ability of cell lines to provide abundant biological material allows researchers to experimentally measure out the sensitivity of the cell line to many different drug and therapy options. Also, since the cell lines and any previously obtained properties or experimental results are being made publicly available and therefore shared between countless laboratories worldwide, analyses do not need to be repeated for every lab or research question. This makes collaboration between different researchers and labs much easier. Due to the advances in high-throughput sequencing and the future tendency to select treatment strategies for every patient based on similarities of that specific tumors molecular and cellular profile to a known cancer cell cine, a large and wide encompassing library of cancer cell lines will only become more important in the future~\cite{cancer_cell_lines_useful_model}.

As actual tailor-made medicine for cancer treatment is still rather far away from being applicable every time in practice, we want to focus on less helpful, but more achievable alternative methods. One of those would be to be able to predict the efficacy of different cancer drugs for a specific cancer patient. One strategy to achieve this goal, known as \textit{drug sensitivity prediction}, is to use cancer cell lines and the data readily available about them to find statistical associations between molecular properties of one specific cell line and drug sensitivity values for that cell line.%\rewrite{maybe expand}

Drug sensitivity prediction for cancer drugs is currently a very present topic in clinical oncology. It is the foundation for creating personalized cancer medicine based on the characterizations of one specific tumor, but also a tool which can help intermediately in the selection of the most appropriate drug from the pool of currently available cancer drugs. Different approaches have been implemented already, as summarized in the \emph{NCI-DREAM drug sensitivity prediction challenge}~\cite{nci_dream}. This project also aims to assess and classify the performance of the different algorithms, analyzing a total of 44 algorithms. The study found that the best algorithms modeled nonlinear relationships and provided the option to incorporate additional information, primarily about biological pathways.
%like random forest algorithms~\cite{predicting_drug_sensitivity_random_forests}, deep learning methods~\cite{predicting_drug_sensitivity_deep_learning} or combined machine learning approaches such as kernelized Bayesian multitask learning~\cite{predicting_drug_sensitivity_bayes}\addref{cite 1 review over the entire field instead of three individual examples for methods. DREAM Drug Sensitivity Prediction Challenge}


%\textsc{Simplify the question and lead into statistical association between cellular properties of the cancer cell lines and drug sensitivity data}\\
%\color{Gray}While those methods might lead to important and helpful results, they tend to be difficult to apply in practice\todo{is this even true?}\addref{difficult to apply}.\color{Black}\\
In this thesis, we will perform the first steps to a simplified version of this prediction: we want to use available data, like in our case biological properties about the cancer cell lines (such as commonly characterized gene expression alterations) and drug sensitivity data (available through e.g. the \textsc{GDSC} database), and run a variety of statistical tests on that data, including the use of \textbf{GeneTrail}~\cite{genetrail} to perform \textit{Gene Set Enrichment Analysis}. That way, we can attempt to find statistical associations between properties of a specific cell line and a noteworthy sensitivity or resistance to a specific drug, and use those findings to give simple, but helpful aid in drug sensitivity prediction.%\rewrite{check}