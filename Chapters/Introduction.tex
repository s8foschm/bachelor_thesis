%!TEX root = ../Thesis.tex
Cancer is a very challenging family of diseases which has been at the forefront of biological and biomedical research for many years now. It is characterized by abnormal cell growth and multiplication, forming clumps of cells called \textit{tumors}, which can spread across the body in a process called metastasis, which often leads to death \cite{who-cancer}. It has been a leading cause of death in Germany and all over the world for many years, being the second most important cause of death globally in 2023 \cite{owid-causes-of-death}. Still, scientists lack a profound understanding of the causes, development and consequences of cancer, and comprehensive success in cancer therapy has been achieved yet, as most current treatments remain highly dangerous, invasive, or their success rate very uncertain \cite{side-effects-cancer-treatments, who-cancer} \addref{danger+uncertainty of treatments}.\\~\\
Often, cancer formation is triggered by mutations in the DNA of those cells. These can either be inherited or happen during the cell's lifespan. Certain substances or exposures, known as \textit{carcinogens}, such as for example tobacco, UV radiation, different viruses or some types of radiations, can greatly increase the probability of a cell developing cancer \cite{carcinogens}.\\~\\
One main challenge is that cancer cells and their properties tend to differ strongly between different tumors, since their properties are being caused by mutations, which are inherently random. Therefore, there is a vast and diverse array of mutations, that can occur in very varied cases. Still, some similarities and common traits can be identified between different tumors, as first done in 2000 in \textit{The Hallmarks of Cancer} by Hanahan and Weinberg \cite{hallmarks-of-cancer} and later extended in 2011, now comprising ten characteristics \cite{hallmarks-of-cancer-next-generation}.\\
These specific cancer-facilitating mutations can be cataloged into different fundamental ways of function, each bypassing one defense mechanism deeply engrained into the fundamental biology of each and every cell. 6 fundamental hallmarks were identified, as well as two emerging hallmarks and two enabling characteristics, which are suspected to lead to the development of the proposed hallmarks.\\
The first of these hallmarks is that cancer cells are self-sufficient in growth signals. That means that they don't require signals (by way of hormones, e.g.) to initiate cell growth; they either produce these signals themselves (through so-called \textit{autocrine signaling}), activate the growth pathway (that would usually be activated by the previously mentioned signals), or destroy the negative feedback loops in place to prevent excessive growth through these signals.\\
The second and third hallmarks are that cancer cells tend to be insensitive to anti-growth signals and apoptosis. Similarly to the previous hallmark, the cancer cell circumvents the typical mechanisms meant to avoid uncontrolled growth and abnormal proliferation of cells; or, in the case of apoptosis, the most drastic and last line of defense against abnormally proliferating cells, the induced cell death.\\
The next two hallmarks are related to the actual uncontrolled growth and multiplication of tumors and cancer cells. They are, in fact, the ability of cancer cells to replicate an unlimited number of times (whereas regular cells die after a certain number of divisions), as well as the sustained formation of new blood vessels (\textit{angiogenesis}) to ensure the needed supply of oxygen and nutrients for the growth of a tumor.\\
Finally, the last hallmark of cancer is the ability of cancer cells to break away from their original site or organ of origin to invade other surrounding tissue and spread all across the body in a process called \textit{metastasis}. This is the most known consequence of cancer, and extremely dangerous for the health and survival of the patient, as it can lead to failure in any organ that it might reach, and metastatic cancer (also known as \textit{Stage IV}) can currently not be treated.\\~\\
The unpredictability \cite{heterogeneity-implications-targeted-therapeutics} in how a certain tumor or cancer patient will react to specific cancer drug is caused by the significant heterogeneity that can be found in cancer, which is in turn caused by the different random mutations present in every cancer cell. They can differ in multiple aspects between different patients (\textit{inter-patient heterogeneity}), between different tumors of the same patient (\textit{inter-tumor heterogeneity}) or even between individual cells within the same tumor (\textit{intra-tumor heterogeneity}). \cite{heterogeneity-implications-targeted-therapeutics}\\~\\
One of the problems caused by this heterogeneity is that it makes the way a certain tumor or patient would react to a specific cancer drug very unpredictable. A drug which is very effective on one patient might not do anything for another patient with the same kind of tumor. It is a factor which significantly limits the efficacy of targeted therapies currently \cite{heterogeneity-is-it-a-problem}, which makes it a very interesting topic of research for optimizing the ways we treat cancer patients. Clinically, the frequency at which cancer cells evolve and change, as well as the intra-tumor heterogeneity pose a large challenge, since the differences between individual cells inside a tumor, as well as the rapidly changing properties might induce a significant difference between the cells on which the genetic profiling has been performed after they were obtained from a biopsy, and the cells which are in the tumor at the time of the therapy \cite{heterogeneity-is-it-a-problem}. That difference makes the efficacy of a treatment unpredictable in many cases.\\~\\
\textsc{insert paragraph about typical current cancer therapy here.}\\~\\
Personalized cancer therapy is one of the main goals of cancer research, since it would allow medical practitioners to administer drugs with a higher probability of success and a lower chance of unwanted side effects \cite{meric2012overcoming}. The dream scenario would be the ability to produce custom-made drugs for every patient, but that is still not available for all types and subtypes of cancer \cite{personalized_cancer_medicine}.\rewrite{expand}\\~\\
Due to the very significant variance and heterogeneity observed in natural cancer cells, a model system is needed for research purposes. A very commonly used in vitro model system in cancer research are so-called cancer cell lines. These consist of highly reproducible cancer cells kept in a lab setting, with very well-known properties \cite{cancer_cell_line_definition}. Cancer Cell Lines are very useful because they are very well-characterized with omics technologies (i.e.\ genomics, transcriptomics, and proteomics) \cite{cancer_cell_lines_useful_model}.\rewrite{expand}\\~\\
As actual tailor-made medicine for cancer treatment is still rather far away from being applicable in practice, we want to focus on less helpful, but more achievable alternative methods. One of those would be to be able to predict the efficacy of different cancer drugs for a specific cancer patient. One strategy to achieve this goal, known as \textit{drug sensitivity prediction}, is to use Cancer Cell Lines and the data readily available about them to find statistical associations between molecular properties of one specific cell line and drug sensitivity values for that cell line.\rewrite{expand}\\~\\