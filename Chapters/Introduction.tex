%!TEX root = ../Thesis.tex
Cancer is a very challenging family of diseases which has been at the forefront of biological and biomedical research for many years now. It is characterized by abnormal cell growth and multiplication, forming so-called tumors which can spread across the body in a process called metastasis, which often leads to death \citep{who-cancer}. It has been a leading cause of death in Germany and all over the world for many years, being the second most important cause of death globally in 2023 \citep{owid-causes-of-death}. Still, scientists lack a profound understanding of the causes, development and consequences of cancer, and comprehensive success in cancer therapy has been achieved yet, as most current treatments remain highly dangerous, invasive, or their success rate very uncertain\addref[danger+uncertainty of treatments].\\
That is also caused by the significant heterogeneity that can be found in cancer: cancer cells can differ in multiple aspects between different patients (\textit{inter-patient heterogeneity}), between different tumors of the same patient (\textit{inter-tumor heterogeneity}) or even between individual cells within the same tumor (\textit{intra-tumor heterogeneity}).\addref[heterogeneity]\\
One of the problems caused by this heterogeneity is that it makes the way a certain tumor or patient would react to a specific cancer drug very unpredictable. A drug which is very effective on one patient might not do anything for another patient with the same kind of tumor.\rewrite[]\\
Personalized cancer therapy is one of the main goals of cancer research, since it would allow medical practitioners to administer drugs with a higher probability of success and a lower chance of unwanted side effects \citep{meric2012overcoming}. The dream scenario would be the ability to produce custom-made drugs for every patient, but that is still not available for all types and subtypes of cancer \citep{personalized_cancer_medicine}.\\
Due to the very significant variance and heterogeneity observed in natural cancer cells, a model system is needed for research purposes. A very commonly used in vitro model system in cancer research are so-called cancer cell lines. These consist of highly reproducible cancer cells kept in a lab setting, with very well-known properties \citep{cancer_cell_line_definition}. Cancer Cell Lines are very useful because they are very well-characterized with omics technologies (i.e. genomics, transcriptomics, and proteomics) \citep{cancer_cell_lines_useful_model}.\\
As actual tailor-made medicine for cancer treatment is still rather far away from being applicable in practice, we want to focus on less helpful, but more achievable alternative methods. One of those would be to be able to predict the efficacy of different cancer drugs for a specific cancer patient. One strategy to achieve this goal, known as \textit{drug sensitivity prediction}, is to use Cancer Cell Lines and the data readily available about them to find statistical associations between molecular properties of one specific cell line and drug sensitivity values for that cell line.