%!TEX root = ../Thesis.tex
Cancer is a very challenging family of diseases which has been at the forefront of biological and biomedical research for many years now. It is characterized by abnormal cell growth and multiplication, forming clumps of cells called \textit{tumors}, which can spread and invade different parts of the body and damage organs, and often lead to death~\cite{who-cancer}. It has been a leading cause of death \color{Gray}in Germany and \color{Black}all over the world for many years, being the second most important cause of death globally in 2023~\cite{owid-causes-of-death}. Still, scientists lack a profound understanding of the causes, development and consequences of cancer, and comprehensive success in cancer therapy has been achieved yet, as most current treatments remain highly dangerous, invasive, or their success rate very uncertain~\cite{side-effects-cancer-treatments, narrative-review-heterogeneity-challenges}.
% side-effects-cancer-treatments: yale medicine fact sheet about side effects
% narrative-review-heterogeneity-challenges: paper on ncbi about heterogeneity, highlighting (among other things) the unpredictability

Often, cancer formation is triggered by mutations in the DNA of those cells. These can either be inherited or happen during the cell's lifespan. Certain substances or exposures, known as \textit{carcinogens}, such as for example tobacco, different viruses or some types of radiations, can greatly increase the probability of a cell developing cancer~\cite{carcinogens}. Other common carcinogens include obesity, alcohol consumption, poor diet or lack of physical activity. In summary, the probability of developing cancer can be reduced by avoiding known risk factors, but still, the formation of the disease is random and can therefore never be completely ruled out.

One main challenge is that cancer cells and their properties tend to differ strongly between different tumors, since their properties are being caused by mutations, which are inherently random. Therefore, there is a vast and diverse array of mutations, that can occur in a very varied manner between different tumors or patients. Still, some similarities and common traits can be identified between different tumors, as first done in 2000 in \textit{The Hallmarks of Cancer} by Hanahan and Weinberg~\cite{hallmarks-of-cancer} and later extended in 2011, now comprising ten characteristics that are hypothesized to be common among all human cancers~\cite{hallmarks-of-cancer-next-generation}.\\
These specific cancer-facilitating modifications can be cataloged into different fundamental ways of function, each bypassing one defense mechanism deeply engrained into the fundamental biology of each and every cell. 6 fundamental hallmarks were identified, as well as two emerging hallmarks and two enabling characteristics, which are suspected to lead to the development of the proposed hallmarks.

The unpredictability in how a certain tumor or cancer patient will react to specific cancer drug~\cite{heterogeneity-implications-targeted-therapeutics} is caused by the significant heterogeneity that can be found in cancer, which is in turn caused by the different random mutations present in every cancer cell. They can differ in multiple aspects between different patients (\textit{inter-patient heterogeneity}), between different tumors of the same patient (\textit{inter-tumor heterogeneity}) or even between individual cells within the same tumor (\textit{intra-tumor heterogeneity})~\cite{heterogeneity-implications-targeted-therapeutics}.

One of the problems caused by this heterogeneity is that it makes the way a certain tumor or patient would react to a specific cancer drug very unpredictable. A drug which has clinically proven to be very effective on one patient might not do anything for another patient with the same kind of tumor. It is a factor which significantly limits the efficacy of targeted therapies currently~\cite{heterogeneity-is-it-a-problem}, which makes it a very interesting topic of research for optimizing the ways we treat cancer patients. Clinically, the frequency at which cancer cells evolve and change, as well as the intra-tumor heterogeneity pose a large challenge: Since these might induce a significant difference between the cells on which the genetic profiling has been performed after they were obtained from a biopsy, and the cells which are in the tumor at the time of the therapy~\cite{heterogeneity-is-it-a-problem}\rewrite{clarify}. That difference makes the efficacy of a treatment unpredictable in many cases~\cite{narrative-review-heterogeneity-challenges}.

Current\topic{Current methods} methods for treatment of cancer are very varied. One common therapy is using chemotherapy (toxic compounds that kill cancer cells) or radiation therapy (using x-ray, gamma rays or similar types of radiation to kill cancer cells). The problem is that these methods are very aggressive and risky, since they are unable to filter for cancer cells and also attack regular, healthy cells, making them very taxing on the health of the patient. Chemotherapy is the most commonly used therapy against cancer currently, but it often has a drastic impact on the patients health and quality of life, as it can cause many side effects such as poor appetite due to nausea and vomiting, pain, fatigue, as well as psychological symptoms caused by aspects such as chemotherapy-induced hair loss~\cite{cancer-chemotherapy-and-beyond}\rewrite{expand}.

Personalized\topic{Personalized therapy} cancer therapy is one of the main goals of cancer research, since it would allow medical practitioners to administer drugs with a higher probability of success and a lower chance of unwanted side effects~\cite{meric2012overcoming}. The dream scenario would be the ability to produce custom-made drugs for every patient, but that is still not available for all types and subtypes of cancer~\cite{personalized_cancer_medicine}.\\
In recent years, there has been a shift from an organ-centric approach to cancer treatment to a personalized, patient-specific approach, based on deep molecular analysis~\cite{personlized-medicine-recent-progress-cancer}. This has been one of the most important advances in modern oncology~\cite{personlized-medicine-recent-progress-cancer}. Currently, molecular profiling is increasingly being used as a guiding principle for cancer therapy, but the current landscape of the scientific literature is vast and overwhelming. Tools like the \textit{Personalized Cancer Therapy} Knowledge Base~\cite{personalized-cancer-therapy} are being created for that purpose; but still, the necessary information for the specific situation of a patient might be difficult to find or unavailable. Biomarker-guided medicinal product development has been included in the most 6th and most recent revision of the \textit{Guideline on the clinical evaluation of anticancer medicinal products}~\cite{ema_guideline}, published by the European Medicines Agency in January 2024. Still, the processes of changing official regulations and recommendations can often take a very long time, which is why the previously mentioned \textit{Personalized Cancer Therapy} Knowledge Base aims to provide scientific support for the implementation of personalized cancer therapy beyond the official regulations and recommendations~\cite{personalized-cancer-therapy}.

Due\topic{heterogeneity - cancer cell lines} to the very significant variance and heterogeneity observed in natural cancer cells, a model system is needed for research purposes. A very commonly used in vitro model system in cancer research are so-called cancer cell lines. These consist of highly reproducible cancer cells kept in a lab setting, with very well-known properties~\cite{cancer_cell_line_definition}. Cancer Cell Lines are very useful because they are very well-characterized with omics technologies (i.e.\ genomics, transcriptomics, and proteomics)~\cite{cancer_cell_lines_useful_model}. They are able to provide an indefinite source of biological material for experimental purposes~\cite{cancer_cell_lines_useful_model}, which allows for experiments to be reproducible. Also, since the cell lines and any previously obtained properties or experimental results are publicly available and therefore shared between countless laboratories worldwide, analyses do not need to be repeated for every lab or research question. This makes collaboration between different researchers and labs much easier. Due to the advances in high-throughput sequencing and the future tendency to select treatment strategies for every patient based on similarities of that specific tumors molecular and cellular profile, a large and wide encompassing library of Cancer Cell Lines will only become more important in the future~\cite{cancer_cell_lines_useful_model}.

As\topic{tailor-made treatment, drug sensitivity prediction} actual tailor-made medicine for cancer treatment is still rather far away from being applicable every time in practice, we want to focus on less helpful, but more achievable alternative methods. One of those would be to be able to predict the efficacy of different cancer drugs for a specific cancer patient. One strategy to achieve this goal, known as \textit{drug sensitivity prediction}, is to use Cancer Cell Lines and the data readily available about them to find statistical associations between molecular properties of one specific cell line and drug sensitivity values for that cell line.\rewrite{expand}

Drug sensitivity prediction for cancer drugs is currently a very present topic in clinical oncology. It is the foundation for creating personalized cancer medicine based on the characterizations of one specific tumor, but also a tool which can help intermediately in the selection of the most appropriate drug from the pool of currently available cancer drugs. Different approaches have been implemented already, like random forest algorithms~\cite{predicting_drug_sensitivity_random_forests}, deep learning methods~\cite{predicting_drug_sensitivity_deep_learning} or combined machine learning approaches such as kernelized Bayesian multitask learning~\cite{predicting_drug_sensitivity_bayes}\rewrite{maybe expand}.

\textsc{Simplify the question and lead into statistical association between cellular properties of the cancer cell lines and drug sensitivity data}